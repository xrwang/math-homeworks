\documentclass[11pt]{article}
% decent example of doing mathematics and proofs in LaTeX.
% An Incredible degree of information can be found at
% http://en.wikibooks.org/wiki/LaTeX/Mathematics

% Use wide margins, but not quite so wide as fullpage.sty
\oddsidemargin 0.25in
\evensidemargin 0.25in
\topmargin 0.25in
\textwidth 6.5in \textheight 8 in
% That's about enough definitions

\usepackage[fleqn]{amsmath}
\usepackage{upgreek}

\begin{document}
\author{Xiaowei Wang}
\title{\vspace{-3cm}HW 3}
\maketitle

\begin{enumerate}

\item
 \begin{enumerate}
 \item [a.] The expected value for the number of students in the sample older than 25 is:
    \begin{flalign*}
        E(x\geq 25) = 200.
        E(x\geq 25) = \frac{160+240}{2}
    \end{flalign*}
 The SE for the number of students in the same older than 25 = 40.

    \begin{flalign*}
        \frac{ (160-200)^2 + (240-200)^2   }{2} = 1600\\
        SD = \sqrt[2]{1600} \\
        SE = \frac{40}{\sqrt[2]{2} } = 28.36
    \end{flalign*}

     \item [b.]
        The expected value for percentage of students older than 25 :
        \begin{flalign*}
            E(X)  = p, E(x) = 0.4. \\
            SE = \frac{\sqrt[2]{0.4(1-.4)}}{\sqrt[2]{400} } = 0.024
        \end{flalign*}
    \end{enumerate}
\item
    If $n=25$ and the expected value is 50, the SE is 10, if $n=100$, the expected value is 200 $ (50 \times 4) $and the SE is $ 5 $
            \begin{flalign*}
                SE = \frac{SD}{\sqrt[2]{n}} \\
                10 = \frac{SD}{\sqrt[2]{25}} \\
                SD = 50 \\
                SE of 100 = \frac{50}{\sqrt[2]{100}}
            \end{flalign*}


\item
    \begin{enumerate}
            \item [a.] true, the mean of 0,2,3,4 and 6 is 3. For 100 draws, $ 100 \times 3 = 300 $
            \item [b.] false (?)
            \item [c.] false, we can't say with certainty
            \item [d.] false, can't say with certainty
    \end{enumerate}

\item
        SE of N=12 is 10 \\
        \begin{flalign*}
            SE = \frac{SD}{\sqrt[2]{n}} \\
            10 = \frac{SD}{\sqrt[2]{12}} \\
            SD = 34.641 \\
                 \end{flalign*}

            SE of N =22 is: \\
        \begin{flalign*}
            \frac{34.641}{\sqrt[2]{22}} = 7.385 \\
         \end{flalign*}


\item
            SE of N=25 is 50 \\
        \begin{flalign*}
            SE = \frac{SD}{\sqrt[2]{n}} \\
            50 = \frac{SD}{\sqrt[2]{25}} \\
            SD = 250 \\
                 \end{flalign*}

            SE of N =64 is: \\
        \begin{flalign*}
            \frac{250}{\sqrt[2]{64}} = 31.25 \\
         \end{flalign*}


\item 0 (?). It can approach 0.5 but it won't be exactly 0.5 (?)

\item
    The correlation between the heights would be 0 (none of the heights are related to each other).
\item
    nb: used an excel spreadsheet and the formula to calculate covariance
    \item [a.] -1.6
    \item [b.] 0.3
    \item [c.] 2

\end{enumerate}
\end{document}
