\documentclass[12pt]{article}
\usepackage[utf8]{inputenc}
\usepackage{fancyhdr}
\usepackage{extramarks}
\usepackage{amsmath}
\usepackage{amsthm}
\usepackage{amsfonts}
\usepackage{tikz}
\usepackage[plain]{algorithm}
\usepackage{algpseudocode}

\usetikzlibrary{automata,positioning}
%\usepackage{layout}

% \setlength{\voffset}{-0.75in}
% \setlength{\headsep}{5pt}

\title{\vspace{-5.5cm}HW 2}
\author{Xiaowei Wang}
\date{August 8 2017}



\begin{document}
\maketitle


\begin{enumerate}
    \item Using Bayes Rule:
    \\ $P(H_{1}\mid O_{1})= \frac {P(O_{1}\mid H_{1})P(H_{1})}
    {P(O_{1}\mid H_{1})P(H_{1}) + P(O_{1}\mid\neg H_{1})P(\neg H_{1}) }$
    \\
    \\ $= \frac{ \frac{1}{3}\times \frac{1}{2}}
    {\frac{1}{3}\times \frac{1}{2}+ (\frac{2}{3} \times \frac{1}{2}) }$
    \\
    \\ $= 0.333 $

    \item
        \begin{tabular}{ L  L  L }
            $-2$ & 0 & 2 \\
            \hline\hline
            $\frac{1}{6}$ & $\frac{1}{6}$ & $\frac{1}{6}$
        \end{tabular}
        \\
        \text The expected value:
    \\$ E(\textit{X}) = -2 \times \frac{1}{6} + 0 \times \frac{1}{6} + 2 \times \frac{4}{6} = 1     $
    \\
    \\ Variance:
    \\ $ V(X) = (-2-1)^{2} \times \frac{1}{6} + (0-1)^{2} \times \frac{1}{6} + (2-1)^{2} \times \frac{4}{6} = \frac{14}{6}  $

    \item
        \begin{enumerate}
            \item [a.]
            $ \int_{-1}^{1} h(1-z^2) dz = 1 \\
            h(z- \frac{1}{3}z^3) \rvert_{-1}^{1} = 1 \\
            h [1- -\frac{1}{3} \times 1^3 ] - h [-1- -\frac{1}{3} \times (-1)^3 ] \\
            h[\frac{2}{3} + \frac{2}{3}] =1\\
            \frac{4}{3}h = 1,  h = \frac{3}{4}$ \\
            \item[b.]
            CDF is the integral of the PDF.\\
            PDF is $\frac{3}{4}(1-z^2)$ \\
            CDF is $   \int_{-1}^{z}\frac{3}{4}(1-z^2)   $ \\
            \item [c.]
            The probability between 0 and 0.5: \\
            $
            \frac{3}{4}(z- \frac{1}{3}z^3) \rvert_{0}^{0.5} = 1 \\
            \frac{3}{4} [0.5- -\frac{1}{3} \times 0.5^3 ] - \frac{3}{4} [0-\frac{1}{3} \times (0)^3 ] \\
            P(0 \leq p \leq 0.5) = 0.34375
            $

        \end{enumerate}


    \item
        \begin{enumerate}

        \item [a.] P(second ticket is 3 without knowing the first) $ = p, 0\leq p \leq 1  $
        \\ The probability that the second ticket is 3, given that the first ticket is 2 is $\frac{1}{5}$
        \item [b.] The unconditional probability is different than the conditional, because the value of the second ticket is dependent on the first ticket.
        \item [c.] 2 and 3: $ P(2 \cap 3) = \frac{1}{6} \times \frac{1}{5} = \frac{1}{30} $
        \\ probability if first or second is a 3:
        $ \frac{1}{6} and \frac{1}{5}, since the first is unconditional and then the second probability is the first card is removed  $

        \end{enumerate}

    \item
        \begin{enumerate}
            \item [a.]
            The mean of the box is 0.5. Get the mean and then put in how much each value deviates from the mean, which is -0.5 (0-0.5) and 0.5 (1-0.5). \\
            $ SD = \sqrt[2] { \frac{-0.5^{2} + 0.5^{2}  }{2} } = 0.5$
            \item [b.]
            The variance is the Standard Deviation squared, $ 0.5^{2} = 0.25$
            \item [c.]
            The average of n tickets is not a random variable \\
            The variance is $ n \times$ the variance found above $0.25\textit{n} $ \\
            The SD is $ n \times$ the SD found above, $0.5\textit{n}$
        \end{enumerate}

    \item
        \begin{enumerate}
            \item [a.]
            \begin{tabular}{|a | a | a|}
            a & b & avg \\
            \hline\hline
            1 & 2 & 1.5\\
            1 & 3 & 2 \\
            2 & 3 & 2.5 \\
            2 & 1 & 1.5 \\
            3 & 1 & 2 \\
            3 & 2 & 2.5 \\
            \hline
             & avg of avg & 2 \\
             \hline
             & box avg & $ \frac{1+2+3}{3} = 2 $ \\
            \hline\hline
            \end{tabular}
            \item [b.]
            The average of all the first values is 2
            \item [c.]
            The average of the second values is also 2
            \item [d.]
            Each of these possible outcomes are equally likely
            \item [e.]
            The expected value of the first draw is not equal to the expected value of the second. Something about moment of a distribution (?)


        \end{enumerate}

\end{enumerate}


\end{document}
